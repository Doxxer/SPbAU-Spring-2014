\documentclass[russian]{article}
\usepackage[T1]{fontenc}
\usepackage[utf8]{inputenc}
\usepackage{geometry}
\geometry{verbose,tmargin=2cm,bmargin=2cm,lmargin=1cm,rmargin=1cm}
\usepackage{amsmath}
\usepackage{float}
\usepackage{textcomp}
\usepackage{amssymb}
\usepackage{graphicx}
\usepackage{babel}
\usepackage{mathtools}
\usepackage[T2A]{fontenc}
\makeatletter
\@ifundefined{date}{}{\date{}}
\begin{document}

\title{Алгоритмы. HW\#12}
\author{Тураев Тимур, 504 (SE)}
\maketitle

\paragraph*{1}

\textit{Найти рефрен}

Построим по строке сжатое суффиксное дерево. Легко доказать, что в дереве будет линейное число внутренних вершин (листьев ровно $n+1$, а каждая из внутренних вершин имеет по крайней мере 2 потомка, то есть прибавляет к своему поддереву по крайней мере 1 лист. Значит всего вершин -- $O(n)$).

Запустим DFS, посчитаем число листьев в для каждой вершины -- это число будет означать число вхождений строки, ассоциированной с вершиной. 

Ну и все, задача решена -- для каждой строки (понятно что на ребрах останавливаться нет смысла, всегда можно увеличить подстроку не изменив число вхождений пройдя до ближайшей вернины) мы знаем ее длину и число вхождений. Запускаем DFS, считаем максимум и выдает ответ.

Время, понятно, линейно.

\paragraph*{3}

\textit{Призрак Вася.}

Для наглядности рассмотрим строку, которую может видеть Вася: $aaaaxq$. Возможные исходные строки: $qxa\textbf{a}a$ или $q\textbf{x}aa$.

Теперь ясно, что задачу можно переформулировать так: найти в заданной строке все префиксы-палиндромы четной длины. Это и будут концы исходной строки и отражение в зеркале. По этим данным легко восстановить все возможные исходные строки.

Как найти все такие префиксы? Легко проверить (и доказать тоже легко), что, если префикс длины $k$ равен суффиксу той же длины у реверснутой строки, то этот префикс является палиндромом. 

Отсюда решение: разворачиваем строку, строим по ней суффиксное дерево, дальше идем по всем четным префиксам исходной строки и спускаемся по соответствующим символам в дереве. Если остановились в какой-то вершине, в которой заканчивается какой-то суффикс, то мы нашли тот самый префикс, равный суффиксу перевернутой строки -- значит он палиндром. Продолжаем идти по ветке (и четным префиксам) дальше, пока можно, чтобы найти все возможные слова.

\end{document}
