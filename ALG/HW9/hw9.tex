\documentclass[russian]{article}
\usepackage[T1]{fontenc}
\usepackage[utf8]{inputenc}
\usepackage{geometry}
\geometry{verbose,tmargin=2cm,bmargin=2cm,lmargin=1cm,rmargin=1cm}
\usepackage{amsmath}
\usepackage{float}
\usepackage{textcomp}
\usepackage{amssymb}
\usepackage{graphicx}
\usepackage{babel}
\usepackage{mathtools}
\usepackage[T2A]{fontenc}
\makeatletter
\@ifundefined{date}{}{\date{}}
\begin{document}

\title{Алгоритмы. HW\#9}
\author{Тураев Тимур, 504 (SE)}
\maketitle

\paragraph*{1}

\textit{Найти сумму Минковского}

Пусть $A = \{a_1, a_2, \ldots, a_m \}$, $B = \{b_1, b_2, \ldots, b_k \}$
Тогда можно представить $A$ как многочлен степени не выше $10n$: $x^{a_1} + x^{a_2} + \ldots + x^{a_m}$. Аналогично $B$ представляем в виде $x^{b_1} + x^{b_2} + \ldots + x^{b_k}$.

Перемножив эти многочлены получим: $g_{1,1} \cdot x^{a_1 + b_1} + \ldots + g_{m,k} \cdot x^{a_m + b_k}$. Достаточно очевидно, что искомое множество $C$ есть все показатели степеней в получившемся многочлене. А заодно мы автоматически получили и ответ на вопрос: а скольими способами можно представить число $a_i + b_j$ в виде суммы двух элементов двух множеств -- число способв равно коэффициенту перед соответствующей степенью, то есть $g_{i, j}$. 

Так как мы умеем перемножать многочлены степени не выше $10n$ за $O(10n \cdot \log(10n)$, то искомое множество можно найти за $O(n\log n)$

\paragraph*{2}

\textit{По заданным комплексным $z_i$ и неотрицательным целым $a_i$ посчитайте коэффициенты полинома $\prod_i(x-z_i)^{a_i}$. Пусть $n = \sum a_i$. Решите задачу за $O(n \log n)$}

\begin{enumerate}
\item[1.] Научимся сначала считать полином $(x-k)^n$ достаточно быстро.

Если $n$ четно, то $(x-k)^n = ((x-k)^{n/2})^2$ и можно составить рекуррентное соотношение $T(n) = T(n/2) + O(n/2 \cdot \log (n/2))$ -- считаем сначала полином в степени в 2 раза меньше, потом перемножаем его с собой -- так как он степени $n/2$ то это мы можем сделать за $O(n/2 \cdot \log (n/2))$. Решаем по master-method, получаем что $T(n) = O(n \cdot \log n)$

Если $n$ нечетно, то $(x-k)^n = ((x-k)^{n/2})^2 \cdot (x-k)$. В квадрат, как мы уже знаем, можно возвести за $O(n \log n)$. Но понятно, что зная коэффициенты $((x-k)^{n/2})^2$, умножить на $(x-k)$ достаточно просто: это почленная сумма сдвинутого вектора координат вправо на 1 плюс того же вектора, но каждый коэффициент еще умножен на $-k$. Это можно сделать за $O(n)$. Итог -- те же $O(n \log n)$.

\item[2.] Давайте представим исходное произведение в виде 
\[
(x - z_1)^{a_1} \cdot (x - z_2)^{a_2} \ldots (x - z_k)^{a_k}
\]

Где все $a_i$ упорядочены по возрастанию. Тогда первое произведение мы сделаем за $O(a_2 \log a_2)$, второе за $O(a_3 \log a_3)$ и так далее. В итоге можно оценить это $O(a_2 \log a_2) + O(a_3 \log a_3) + \ldots + O(a_k \log a_k)$ = $O(a_2 \log n) + O(a_3 \log n) + \ldots + O(a_k \log a_k)$ = $O(\log n \cdot (a_2 + a_3 + \ldots + a_k))$ = $O(\log n \cdot (a_1 + a_2 + a_3 + \ldots + a_k))$ = $O(n \cdot \log n)$
\end{enumerate}

\paragraph*{3}

\textit{Про Теплицевы матрицы}

Если коротко, то Теплицева матрица это такая матрица, в которой на диагоналях стоят одинаковые числа, а значит, для ее хранения достаточно $O(n+m)$ памяти: будем хранить лишь <<заголовки>> диагоналей (которых $n+m-1$). То есть всю матрицу можно представить в виде вектора размера $n+m-1$: ($a_{1, m}, a_{1, m-1}, \ldots, a_{1, 1}, a_{2, 1}, \ldots a_{n, 1}$)

Научимся теперь быстро умножать матрицу на вектор $b = (b_1, b_2, \ldots, b_m)^T$. Получится вектор $c = (c_1, c_2, \ldots, c_n)^T$, такой что

\[
c_i = \sum_{j=0}^m{a_{m + i - j} \cdot b_j}
\]

Но этот коэффициент $c_i$ есть ни что иное как коэффициент перед степенью $m+n-1-i$ у многочлена, являющегося результатом произведения многочленов с коэффициентами $a$ и $b$, так как $c_i$ по сути является сверткой.

Ну, а раз мы умеем перемножать многочлены за $O(n \cdot \log n)$, то произведение такой матрицы на вектора работает $O((n+m) \cdot \log (n+m))$

\end{document}
