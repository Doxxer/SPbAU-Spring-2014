\documentclass[russian]{article}
\usepackage[T1]{fontenc}
\usepackage[utf8]{inputenc}
\usepackage{geometry}
\geometry{verbose,tmargin=2cm,bmargin=2cm,lmargin=1cm,rmargin=1cm}
\usepackage{amsmath}
\usepackage{float}
\usepackage{textcomp}
\usepackage{amssymb}
\usepackage{graphicx}
\usepackage{babel}
\usepackage{mathtools}
\usepackage[T2A]{fontenc}
\makeatletter
\@ifundefined{date}{}{\date{}}
\begin{document}

\title{Алгоритмы. HW\#10}
\author{Тураев Тимур, 504 (SE)}
\maketitle

\paragraph*{1}

\textit{Может ли первая команда стать чемпионом}

Построим такую сеть: $<s, t, G, T, U, V, W>$, где $s$ -- это исток, а $t$ -- сток. Множество вершин $G$ -- это всевозможные игры между двумя командами (их число $\binom{n}{2}$). Множество вершин $T$ -- это все команды. 

Ребра $U$ соединяют сток с $G$, причем ребро есть только тогда, когда игры еще не сыграна. Пропускная способность таких ребер -- $1$.

Ребра $V$ соединяют $G$ c $T$: из вершины $g_{i,j}$ выходят 2 ребра в вершины $t_i$ и $t_j$. Пропускная способность таких ребер -- бесконечность. Ясно, что в каждую вершину $t_i$ входит ровно $n-1$ ребро.

Ребра $W$ соединяют вершины $T$ с вершиной $t$, со стоком. Причем пропускная способность ребра $(t_1, t)$ равна бесконечности, а пропускные способности остальных ребер (вида $(t_i, t)$) равны тому числу игр, которая команда $i$ может выигрыть, ничего не испортив при этом для команды 1.

Теперь ясно, что максимальный поток в такой сети даст нужное нам расписание. И ответом на задачу будет <<да>> тогда, когда максимальный поток будет равен числу игр, которые осталось сыграть. Или, другими словами, все ребра выходящие из истока должны быть насыщены.

Остался вопрос как найти пропускную способность ребер $W$. Пусть осталось сыграть $K$ игр (это число легко найти, считывая таблицу). Предположим, первая команда все их выигрывает, тогда максимальное число побед, которые может одержать команда $i$ (то есть пропускная способдность ребра $(t_i, t)$) равно $K - 1 - M$, где $M$ -- число побед, которые уже одержала команда $i$.

\paragraph*{2}

\textit{Задача про сетку}

Хочется, чтобы через каждую вершину проходил максимум один путь, для этого естественно желание повесить пропускную способность не только на ребра, но и на вершины.

Давайте так и сделаем: пусть у каждой вершины $v$ будет какая-то пропускная способность $d$, тогда задача поиска максимального потока в такой сети сводится к обычной следующим образом: разобьем вершину $v$ на 2 обычные вершины: $v_1$ и $v_2$ и проведем между ними ребро с пропускной способностью $d$. Дополнительно, для каждого ребра в исходном графе $(u, v)$ с пропуской способностью $c$ проведем ребро в новом графе вида $(u, v1)$ с тем же весом $c$, аналогично для всех ребер вида $(v, u)$. Достаточно очевидно, что задача поиска максимального потока в исходном графе даст соответствующий максимальный поток в таком графе.

Сведем нашу задачу к такой: пусть сетка это граф, где вершины -- узлы сетка, а ребра -- отрезки сетки. Введем новую вершину $s$ -- исток -- и соединим ее ребрами со всеми выделенными вершинами и вершину $t$ -- сток -- и соединим ее со всеми вершинами на границе, то есть с вершинами степени меньше 4. Далее, как в предыдущем пункте, разобьем каждую вершину на две и назначим всем ребрами в графе вес 1.

Достаточно очевидно, что максимальный поток соответствуют непересекающимся путям в исходном графе, а его величина -- их числу. Если величина потока меньше, чем было отмеченных вершин, то задача не имеет решения.

Размер получившейся сети, очевидно, $O(n \cdot m)$ -- число вершин стало в 2 раза больше (и еще плюс две), а число ребер стало $2|E| + |V|$.

\paragraph*{3}

\textit{Докажите, что после каждой итерации поиска блокирующего потока расстояние между стоком и истоком увеличивается хотя бы на единицу.}

\end{document}
