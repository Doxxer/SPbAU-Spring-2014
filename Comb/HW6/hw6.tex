\documentclass[russian]{article}
\usepackage[T1]{fontenc}
\usepackage[utf8]{inputenc}
\usepackage{geometry}
\geometry{verbose,tmargin=2cm,bmargin=2cm,lmargin=1cm,rmargin=1cm}
\usepackage{amsmath}
\usepackage{float}
\usepackage{textcomp}
\usepackage{amssymb}
\usepackage{graphicx}
\usepackage{babel}
\usepackage{mathtools}
\usepackage[T2A]{fontenc}
\makeatletter
\@ifundefined{date}{}{\date{}}
\begin{document}

\title{Комбинаторика. HW\#6}
\author{Тураев Тимур, 504 (SE)}
\maketitle

\paragraph*{11.5}

\textit{Доказать, что в случае произвольного натурального числа $n$ цикловой индекс группы $D_n$ симметрий правильного $n$-угольника определяется по формуле}

Запишем цикловый индекс по определению:

\[
Z_{D_n}(x_1, \ldots, x_n) = \frac{1}{2n} \sum_{\sigma \in \Sigma_G} x_1^{i_1} x_2^{i_2} \ldots x_n^{i_n} = \frac{1}{2n} \left(\sum_{\sigma \in C_n} x_1^{i_1} x_2^{i_2} \ldots x_n^{i_n} + \sum_{\sigma \in D'_n} x_1^{i_1} x_2^{i_2} \ldots x_n^{i_n} \right)
\] 

Мы разбили сумму на 2 суммы: по всем поворотам и по всем отражениям (и только по ним -- это в формуле выше подгруппа $D'_n$). 

Для первой суммы все известно: упражение 11.4 (оно же -- задание в классе, мы там выяснили что к чему и зачем: проходим по всем делителям $n$ -- это всевозможные длины циклов, а множитель в виде фукнции Эйлера говорит о том сколькими способами мы можем этот цикл получить -- фактически число порождающих элементов циклической группы).

Осталось понять вторую сумму (рассмотрим пока для нечетных $n$). Какое бы мы отражение ни выбрали, схема его всегда одна и та же: линия симметрии проходит через какую-либо вершину и через середину противоположного ребра: эта вершина при отражении образует цикл длины 1, а все остальные -- несколько (а именно $(n-1)/2$) циклов длины 2. Всего таких циклов ровно $n$ (по числу вершин или ребер). Получается, вторая сумма (вместе с множителем перед скобкой) превращается в:

\[
\frac{1}{2n} \cdot n \cdot x_1 x_2^{(x-1)/2} = 1/2 \cdot x_1 x_2^{(x-1)/2}
\]

С четным $n$ все почти аналогично: за исключением того, что там половина циклов одного вида (2 цикла длины 1, а остальные длины 2 -- это когда линия отражения проходит через вершину), а другая половина -- другого (все циклы -- это циклы длины 2 -- это когда линия отражения проходит через середины ребер); но схема подсчета полностью аналогична.

В итоге получим требуемую формулу.

\paragraph*{12.1}

\textit{Доказать, что количество геометрически различных способов окраски ожерелья в не более чем два цвета, при которых ровно $k$ вершин окрашены в один цвет, определяется по формуле}

\[
c_k = \frac{1}{n} \sum_{d | \gcd(n, k)}{\varphi(d) \binom{n/d}{k/d}}
\]

Идея такая же, как в предыдущей задаче: число геометрически различных окрасок это число орбит группы $X$ (группа покрасок ожерелья в 2 цвета с ровно $k$ окрашенными вершинами) под действием группы $G = C_n$. Это число, как мы знаем, можно считать через перечисление циклов (в которые будут переходить геометрически неразличимые покраски ожерелий). 

Итак, у нас есть правильный $n$-угольник, в котором будут помечены каким-то образом $k$ вершин. Какие возможны при этом циклы? Цикл длины 1 есть всегда -- его дает нам нейтральный элемент $C_n$. Когда возможен цикл длины 2? Ясно, что тогда, когда и $k$ делится на 2 (чтобы можно было разбить все $k$ помечаемых точек на 2 равных множества) и $n$ делилось бы на 2: чтобы их можно было <<разложить>> на вершинах симметрично. Аналогично рассуждая, приходим к выводу, что цикл длины 3 возможен когда оба числа делятся на 3. 

Таким образом, возможные циклы описываются всеми общими делителями $n$ и $k$. Или, другими словами, всеми такими $d | \gcd(n, k)$.

Аналогично: число порождающих элементов цикла есть фукнция Эйлера его длины, с этим мы уже дважды разобрались.

Теперь осталась задача лишь выбрать эти $k$ точек. <<Цикл>> длины $d$ разбивает правильный $n$ угольник на $d$ секторов (размером $n/d$), в каждом из которых нам нужно выбрать по $k/d$ окрашенных точек. Но так как мы строим цикл, то ясно, что выбрав какие-то $k/d$ точек в одном секторе, мы однозначно определим выбор в других секторах -- просто <<скопировав>> выбор первого сектора. Таким образом число раскрасок, которые перейдут в себя циклически (с длиной цикла $d$) есть $\varphi(d) \binom{n/d}{k/d}$

Получилась требуемая формула.

\end{document}
