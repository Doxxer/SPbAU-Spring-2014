\documentclass[russian]{article}
\usepackage[T1]{fontenc}
\usepackage[utf8]{inputenc}
\usepackage{geometry}
\geometry{verbose,tmargin=2cm,bmargin=2cm,lmargin=1cm,rmargin=1cm}
\usepackage{amsmath}
\usepackage{float}
\usepackage{textcomp}
\usepackage{amssymb}
\usepackage{graphicx}
\usepackage{babel}
\usepackage{mathtools}
\usepackage[T2A]{fontenc}
\makeatletter
\@ifundefined{date}{}{\date{}}
\begin{document}

\title{Комбинаторика. HW\#2}
\author{Тураев Тимур, 504 (SE)}
\maketitle

\paragraph*{3.8}

\textit{Доказать комбинаторно, что количество способов разбиения натурального числа $n$ на слагаемые, при котором любое число входит в это разбиение не более одного раза, равно количеству разбиений $n$ на нечетные слагаемые}


Существует биекция между разбиением числа на различные слагаемые и разбиением числа на только нечетные слагаемые: каждое четное число в первом разбиении можно разложить в сумму $2^k$ нечетных чисел (просто делим число на 2 пока делится).

В другую сторону: возьмем пару одинаковых чисел и запишем вместо нее их сумму. И так далее, пока все числа не станут различными. 

Почему это будет биекцией?

Ясно, что любое число можно единственным образом представить в виде $2^k \cdot a$, где $a$ -- нечетное число. Поэтому рассмотрим обратное отображение: из суммы нечетных в сумму различных чисел. Предположим, получили 2 разных представления. Это возможно, если в одном случае мы получили какое-то число $2^k \cdot a$, а в другом сумму различных (например двух) четных чисел $2^r \cdot a + 2^q \cdot b$, $r \neq q, r < q < k$. То есть

\[
2^k \cdot a= 2^r \cdot a + 2^q \cdot b
\]
\[
2^k = 2^r + 2^q
\]
\[
2^{k-q} = 2^{r-q} + 1
\]

Слева четное число, справа четным может быть только тогда, когда $r=q$, получили противоречие, значит отображение биективно, значит число таких способов одинаково.


\paragraph*{4.2}

\textit{С использованием диаграмм Ферре показать, что количество разбиений числа $2n+m$ на ровно $n+m$ слагаемых одинаково при любом $m \geqslant 0$}

Рассмотрим диаграмму Ферре разбиения числа $2n+m$ на ровно $n+m$ слагаемых. Заметим, что в самом левом столбце ровно $n+m$ точек. Удалим их. Получим диаграмму Ферре разбиения числа $2n+m - n - m = n$ на не более $n$ слагаемых. Отображение биективно, поэтому искомое число: $P_n(n) = p_n(2n)$ (по формуле 19 конспекта)

\paragraph*{4.6}

\textit{Доказать, что $p(n)^2 < p(n^2 + 2n)$ при $n \geqslant 1$. }

Докажем инъективность отображения.

Возьмем пару любых разбиений числа $n$ на слагаемые (их число -- как раз левая часть неравенства). Теперь сделаем над первым разбиением следующее действие:  приплюсуем справа слагаемые равные нулю так, чтобы общее число слагаемых стало ровно $n$. Затем, прибавим к каждому из $n$ полученных слагаемых число $n$. 


Далее, припишем справа наше второе разбиение. Что получилось? Получилось разбиение числа $n + n^2 + n = n^2 + 2n$ на не менее чем $n$ слагаемых, причем первые $n$ слагаемых не меньше $n$. \textit{Очевидно, это отображение биективно: разобьем слагаемые на 2 группы: в первой $n$ слагаемых, во втором остальные, вычтем из первых $n$ слагаемых $n$, уберем лишние нули -- получим исходную пару разбиений.}

Но также очевидно, что существует еще тонна других разбиений числа $n^2+2n$ в которые переход в принципе невозможен, например, разбиение на меньшее число слагаемых. 

Значит, отображение инъективно.

\paragraph*{4.8}

\textit{Выразить через $p(n)$ количество таких разбиений, в которых равны три наибольшие части.}

Искомое число: $ans = p(n) - A(n) - B(n)$, где $A(n)$ -- число таких разбиений, где наибольшая часть встречается ровно 2 раза, а $B(n)$ -- число таких разбиений, где наибольшая часть встречается ровно 1 раз.


$B(n)$ найти просто -- двойстванная диаграмма Ферре описывает все разбиения числа $n$ где обязательно есть единичка. Удалив ее, получим все разбиения числа  $n-1$. Добавим к каждому -- получим нужное. Значит, $B(n) = p(n-1)$


Что такое $A(n)$? Из двойственной диаграммы ясно, что это такое число разбиений числа $n$, в котором есть хотя бы одна двойка и нет единиц. А как их сосчитать? Давайте уберем эту двойку (которая точно есть) -- получим всевозможные разбиения числа $n-2$ без единиц. А их число мы знаем (это все разбиения в которых две наибольшие части равны), это фактически задача с практики: их число $p(n-2) - p(n-3)$.

Собираем все в одну кучу: $ans = p(n) - p(n-1) - p(n-2) + p(n-3)$


\paragraph*{4.9}

\textit{С использованием диаграмм Ферре показать, что количество разбиений числа $n$ ровно на $k$ частей равно количеству разбиений числа $n+k(k-1)/2$ ровно на $k$ неравных частей.}

Построим диаграмму Ферре разбиения числа $n$ ровно на $k$ частей -- в ней ровно $k$ строк. Добавим в первую строчку $k-1$ точку, во вторую -- $k-2$ и так далее. Получим диаграмму Ферре разбиения числа числа $n+k(k-1)/2$ на ровно $k$ частей. \textit{Очевидно, это отображение биективно: любое разбиение такого числа на $k$ различных частей требует того, чтобы в первой строке нашлась хотя бы $k-1$ точка, во второй -- $k-2$ и так далее.}

Убедимся, что все полученные части различны. Предположим, это не так, нашлись две одинаковые строчки: пусть в верхей из них было $a$ точек, туда добавили $k-i$ точек, а в нижней было $b$ точек, а добавили в нее $k-j$ точек, причем $a \geqslant b, i < j$. Тогда

\[
a + k - i = b + k - j
\]
\[
a - i = b - j
\]
\[
a - b = i - j
\]

Слева число неотрицательное, а справа отрицательное. Противоречие, значит все части различны.

\end{document}
