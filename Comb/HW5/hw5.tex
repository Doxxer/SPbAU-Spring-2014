\documentclass[russian]{article}
\usepackage[T1]{fontenc}
\usepackage[utf8]{inputenc}
\usepackage{geometry}
\geometry{verbose,tmargin=2cm,bmargin=2cm,lmargin=1cm,rmargin=1cm}
\usepackage{amsmath}
\usepackage{float}
\usepackage{textcomp}
\usepackage{amssymb}
\usepackage{graphicx}
\usepackage{babel}
\usepackage{mathtools}
\usepackage[T2A]{fontenc}
\makeatletter
\@ifundefined{date}{}{\date{}}
\begin{document}

\title{Комбинаторика. HW\#5}
\author{Тураев Тимур, 504 (SE)}
\maketitle

\paragraph*{9.1}

\textit{Семь человек садятся за круглый стол. Считается, что способы рассадки этих людей совпадают, если при каждой такой рассадке любой человек имеет около себя одних и тех же соседей. Сколько возможных способов рассадки людей вокруг стола существует?}

\begin{itemize}
\item[1 способ.] Посадим первого человека куда-нибудь. Далее нужно выбрать ему пару соседей, это можно сделать $\binom{n-1}{2}$ способами. Так как порядок рассадки соседенй неважен (по условию задачи), то рассадить их можно одним способом. Далее, на пустое место рядом с одним из соседей мы можем выбрать любого оставшегося человека и получим новую рассадку и так далее, пока все не усядутся.

Итоговая формула

\[
\binom{n-1}{2} \cdot (n-3)! = \frac{(n-1)!}{2} = \frac{6!}{2} = 360
\]
\item[2 способ.] Пусть $n \geqslant 3$ человек уже сидит. Приходит $n+1$-ый человек. Его мы можем посадить в любое из $n$ мест-промежутков, и всегда будет получаться новая рассадка. Отсюда $a_{n+1} = n \cdot a_n$ и $a_3 = 1$. Решаем, получаем $a_n = \frac{(n-1)!}{2}$, $a_7 = 360$. 
\end{itemize}


\paragraph*{10.3}

\textit{Подсчитать с помощью леммы Бернсайда количество геометрически различных способов окраски граней куба в не более чем 7 цветов.}

Тут $|X| = 7^6$.

Группа симметрии куба состоит из 24 элементов.
\begin{itemize}
\item Нейтральный элемент. $|X^e| = |X| = 7^6$

\item 6 90-градусных вращений вокруг оси, проходящей через центры противоположных граней куба. Тут фиксируются цвета граней, через которую провели ось вращения и фиксируется цвет всех остальных четырех граней. Поэтмоу всего есть 7 в кубе (каждую из трех "групп" граней можно покрасить в свой цвет) неподвижных точек. Вклад $6 |X^a| = 6 \cdot 7^3$.

\item 3 180-градусных вращений вокруг оси, проходящей через центры противоположных граней куба. Тут же фиксируются цвета граней, через которую провели ось вращения и фиксируется две группы по 2 грани (противоположные). Поэтмоу всего есть 7 в четвертой степени неподвижных точек. Вклад $3 |X^b| = 3 \cdot 7^4$.

\item 8 120-градусных вращений вокруг главных диагоналей. Тут фиксируется 2 группы граней: инцидентные каждому концу выбранной диагонали. Поэтмоу всего есть 7 в квадрате неподвижных точек. Вклад $8 |X^c| = 8 \cdot 7^2$.

\item 6 180-градусных вращений вокруг оси, проходящей через центры противоположных ребер. Тут фиксируется 3 группы граней, каждую можно покрасить в какой-либо цвет, поэтому вклад $6 |X^d| = 6 \cdot 7^3$.

Ответ:
\[
|X/G| = \frac{1}{24}\left(7^6 + 6 \cdot 7^3 + 3 \cdot 7^4 + 8 \cdot 7^2 + 6 \cdot 7^3 \right) = 5390
\]

\end{itemize}
\end{document}
