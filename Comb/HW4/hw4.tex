\documentclass[russian]{article}
\usepackage[T1]{fontenc}
\usepackage[utf8]{inputenc}
\usepackage{geometry}
\geometry{verbose,tmargin=2cm,bmargin=2cm,lmargin=1cm,rmargin=1cm}
\usepackage{amsmath}
\usepackage{float}
\usepackage{textcomp}
\usepackage{amssymb}
\usepackage{graphicx}
\usepackage{babel}
\usepackage{mathtools}
\usepackage[T2A]{fontenc}
\makeatletter
\@ifundefined{date}{}{\date{}}
\begin{document}

\title{Комбинаторика. HW\#4}
\author{Тураев Тимур, 504 (SE)}
\maketitle

\paragraph*{7.2}

\textit{Вывести рекуррентное соотношение для подсчета всех слабо связных орграфов. Пользуясь им, сосчитать количество таких орграфов, построенных на пяти вершинах.}

Всего простых орграфов $2^{n^2 - n}$. Из них надо вычесть все несвязные орграфы (под ними будем понимать несвязные неориентированные версии ориентированных графов). Действуя прямо по аналогии с началом пункта 7 коспекта получаем, что их число <<похоже>> на 

\[
a_n = 2^{n^2 - n} - \sum_{k=1}^{n-1} \binom{n}{k} \cdot a_k \cdot 2^{(n-k)^2 - (n-k)}
\]

Но это неправда: каждый граф тут будет посчитан несколько раз, причем это число зависит от $k$ и $n$. Сделаем такой трюк: что такое сумма? На нее можно смотреть как на разбиение всех несвязных графов на 2 (упорядоченных) блока -- в первом слабосвязный граф, во втором какой угодно. Давайте помечать <<ведущую>> вершину в левом блоке (в этом блоке лежит слабосвязный граф). Ее мы можем выбрать $k$ способами. Формула становится такой:

\[
a_n = 2^{n^2 - n} - \sum_{k=1}^{n-1} k \cdot \binom{n}{k} \cdot a_k \cdot 2^{(n-k)^2 - (n-k)}
\]

А теперь заметим, что каждый граф (где сумма) мы посчитали ровно $n$ раз: действительно, рассмотрим какой-нибудь несвязный граф, в нем пусть $r$ блоков. Каждый блок будет когда-нибудь <<левым>>, его мы посчитаем ровно $k_{r_i}$ раз, значит всего мы посчитаем этот граф таким числом спобосовв: сумма по всем $k_{r_i}$, а это в точности число $n$. 

И итоговая правильная формула:

\[
a_n = 2^{n^2 - n} - \frac{1}{n} \sum_{k=1}^{n-1} k \cdot \binom{n}{k} \cdot a_k \cdot 2^{(n-k)^2 - (n-k)}
\]

Подсчет дает: $a_1 = 1, a_2 = 3, a_3 = 54, a_4 = 3834, a_5 = 1027080$

\paragraph*{7.4}

\textit{Доказать, что имеется $2^{\lfloor n/2 \rfloor}$ звездных многоугольников на $n$ вершинах.}

Пусть $n$ четное.

После серии рисунков достаточно очевидно, что существуют так называемые <<элементарные графы>>: это такие графы, которые полностью определяются одним ребром. Всего таких графов столько, сколько можно провести различных ребер из какой-нибудь зафиксированной вершины, скажем из вершины 1. Ясно, что всего таких ребер можно провести ровно $n/2$: следующее ребро в вершину с номером $n/2 + 1$ порождате тот же граф, что и ребро в вершину $n/2 - 1$ (ну это ясно из теории чисел и модульной арифметики). Также ясно, что все такие элементарные графы можно друг с другом <<складывать>> и получать новый звездчатый многоугольник: это сработает только тогда, когда все элементарные графы реберно непересекающиеся. Если это так, то искомое число это просто число всех подмножеств множества элементарный графов, то есть $2^{n/2}$.

Вот доказать факт, что элементарные графы не пересекаются уже интереснее. Предположим, это не так, значит у каких-то двух различных элементарных графов совпадает хотя бы 1 ребро. Так как они элементарны, то есть полностью определются одним ребром, то совпадают у них и подграфы, порожденные этим ребром. Пометим эти совпавшие ребра. 

Получили противоречие: либо это все-таки не такие уж и различные графы, либо у какого-нибудь графа есть еще <<непомеченные>> ребра, а тут уже противоречие с тем, что он элементарный.

Таким образом все элементарные графы не пересекаются, значит можно строить их всевозможные реберные объединения.

\textit{Замечание. В случае нечетного $n$ все практически то же самое, только порождающих ребер там будет $(n-1)/2$ или, что то же самое, $\lfloor n/2 \rfloor$}

\paragraph*{7.5}

\textit{Выразить производящую функцию для сильно связных турниров в терминах производящей функции для всех турниров.}

Пусть всего турниров на $n$ вершинах $t_n$, а сильно связных $s_n$. Тогда имеет место следующее рекуррентное соотношение:

\[
t_n = s_n + \sum_{i=1}^{n-1} \binom{n}{i} s_i \cdot t_{n-i} 
\]

Действительно, все турниры это либо один большой сильно связный турнир, либо сильносвязный на $i$ вершинах плюс остальная часть, которая представляет собой просто турнир. Переходя к производящим функциям получим (свертка превращается в произведение):

\[
T(z) = S(z) + S(z) \cdot T(z)
\]

\textit{Вообще говоря, результат можно было получить сразу исходя из смысла сложения и произведения производящих функций, ну да ладно.} Отсюда искомая $S(z)$:

\[
S(z) = \frac{T(z)}{1+T(z)} = 1 - \frac{1}{1+T(z)}
\]

\paragraph*{8.2}

\textit{Перечислить помеченные деревья, вершины которых имеют только степени 1 и 3}

Не, не пошла.

\end{document}
