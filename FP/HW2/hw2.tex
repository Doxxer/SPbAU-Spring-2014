\documentclass[russian]{article}
\usepackage[T1]{fontenc}
\usepackage[utf8]{inputenc}
\usepackage{geometry}
\geometry{verbose,tmargin=2cm,bmargin=2cm,lmargin=1cm,rmargin=1cm}
\usepackage{amsmath}
\usepackage{float}
\usepackage{textcomp}
\usepackage{amssymb}
\usepackage{graphicx}
\usepackage{babel}
\usepackage{mathtools}
\usepackage[T2A]{fontenc}

\makeatletter
\@ifundefined{date}{}{\date{}}

\begin{document}

\title{Функциональное программирование. HW\#2}
\author{Тураев Тимур, 504 (SE)}

\maketitle

\paragraph*{1.}

\textit{Приведите пример замкнутого чистого $\lambda$-терма, находящегося}


\textbf{-- в WHNF, но не в HNF}

Не-редекс на верхнем уровне, но внутри чистый редекс 

\[
\mathtt{\lambda x.(\lambda y.y)x}
\]


\textbf{-- в HNF, но не в NF}

Не-редекс на верхнем уровне, но внутри не чистый редекс, но можно провести $\beta$-редукцию

\[
\mathtt{\lambda x.x((\lambda y.y)x)}
\]


\paragraph*{2.} \textit{Написать $\mathtt{minus}$, $\mathtt{equals}$, $\mathtt{lt}$, $\mathtt{gt}$, $\mathtt{le}$, $\mathtt{ge}$}

\[
\mathtt{minus \equiv \lambda mn.n\;pred\;m}
\]
\[
\mathtt{equals \equiv \lambda mn.AND(iszro(minus\; m\; n))(iszro(minus\; n\; m))}
\]
\[
\mathtt{le \equiv \lambda mn.iszro(minus\; m\; n)}
\]
\[
\mathtt{gt \equiv \lambda mn.NOT(le\; m\; n)}
\]
\[
\mathtt{ge \equiv \lambda mn.OR(gt\; m\; n)(equals\; m\; n)}
\]
\[
\mathtt{lt \equiv \lambda mn.NOT(ge\; m\; n)}
\]


\paragraph*{3.} \textit{Построить функцию $\mathtt{sum}$, суммирующую элементы списка и функцию $\mathtt{length}$, вычисляющую длину списка.}

\[
\mathtt{sum \equiv \lambda l.l(\lambda ht.plus\; h\; t)0}
\]
\[
\mathtt{length \equiv \lambda l.l(\lambda ht.plus\; 1\; t)0}
\]


Пример $\mathtt{sum}$:

\begin{eqnarray*}
sum\; [2, 3] & =\\
(\lambda l.l(\lambda ht.plus\; h\; t)0)\; [2, 3] &=\\
(\lambda l.l(\lambda ht.plus\; h\; t)0)\; (\lambda cn.c\;2\;(c\;3\;n)) &=\\
(\lambda cn.c\;2\;(c\;3\;n))(\lambda ht.plus\; h\; t)0 & =\\
(\lambda ht.plus\; h\; t)\; 2((\lambda ht.plus\; h\; t)\; 3\; 0) &=\\
(\lambda ht.plus\; h\; t)\; 2(plus\; 3\; 0) &=\\
(\lambda ht.plus\; h\; t)\; 2\; 3 &=\\
plus\; 2\; 3 &= &5
\end{eqnarray*}


Пример $\mathtt{length}$. Видно, что это ни что иное, как сумма, которая игнорирует список и складывает единицы:

\begin{eqnarray*}
length [2, 3] & =\\
(\lambda l.l(\lambda ht.plus\; 1\; t)0)\; [2, 3] &=\\
(\lambda l.l(\lambda ht.plus\; 1\; t)0)\; (\lambda cn.c\;2\;(c\;3\;n)) &=\\
(\lambda cn.c\;2\;(c\;3\;n))(\lambda ht.plus\; 1\; t)0 & =\\
(\lambda ht.plus\; 1\; t)\; 2((\lambda ht.plus\; 1\; t)\; 3\; 0) &=\\
(\lambda ht.plus\; 1\; t)\; 2(plus\; 1\; 0) &=\\
(\lambda ht.plus\; 1\; t)\; 2\; 1 &=\\
plus\; 1\; 1 &= &2
\end{eqnarray*}


Дополнение. Более короткие версии:

\[
\mathtt{sum2 \equiv \lambda l.l\;plus\;0}
\]

\[
\mathtt{length2 \equiv \lambda l.l(\lambda ht.succ\; t)0}
\]

Пример на $\mathtt{sum2}$:

\begin{eqnarray*}
sum2\; [2, 3] & =\\
(\lambda l.l\; plus\;0)\; [2, 3] &=\\
(\lambda l.l\; plus\;0)\; (\lambda cn.c\;2\;(c\;3\;n)) &=\\
(\lambda cn.c\;2\;(c\;3\;n))\;plus\;0 & =\\
plus\; 2\;(plus\;3\;0) & = \\
plus\; 2\; 3 &= &5
\end{eqnarray*}


Пример на $\mathtt{length2}$:

\begin{eqnarray*}
length2\; [2, 3] & =\\
(\lambda l.l(\lambda ht.succ\; t)0)\; [2, 3] &=\\
(\lambda l.l(\lambda ht.succ\; t)0)\; (\lambda cn.c\;2\;(c\;3\;n)) &=\\
(\lambda cn.c\;2\;(c\;3\;n))(\lambda ht.succ\; t)\;0 & =\\
(\lambda ht.succ\; t)\; 2((\lambda ht.succ\; t)\; 3\; 0) &=\\
(\lambda ht.succ\; t)\; 2\;1 &= & 2
\end{eqnarray*}



\paragraph*{4.} \textit{Построить функцию $\mathtt{tail}$, возвращающую хвост списка.}

Используем ту же идею, что и с функцией $\mathtt{pred}$.

Определим начальное значение (в $\mathtt{pred}$-е это была пара (0, 0)) -- в нашем случае это будет пара из двух пустых списоков:
\[
\mathtt{xz \equiv pair\; nil\; nil}
\]

Далее, определим функцию, обрабатывающую наш список на каждой итерации. Смысл этой функции вот в чем: мы будем пересобирать список с конца, сохраняя его во втором элементе паре, в первом же элементе будем держать тот же список, но на предыдущей итерации. Таким образом, в конце получим, что в первом элементе паре будет лежать недособранный на один элемент список -- то есть хвост.\\
Этой фунции на вход мы будем подавать очередной элемент списка и пару-состояние: 
\[
\mathtt{fs \equiv \lambda ep.pair\; (snd\; p)\; (cons\; e\; (snd\; p))}
\]

Ну и, наконец, функция $\mathtt{tail}$. Ее поведение уже описано выше.
\[
\mathtt{tail \equiv \lambda l.fst (l\; fs\; xz)}
\]

Пример.

\begin{eqnarray*}
tail\; [5, 3, 2] & =\\
(\lambda l.fst (l\; fs\; xz))\; [5, 3, 2] &=\\
(\lambda l.fst (l\; fs\; xz))\; (\lambda cn.c\;5\;(c\;3\;(c\;2\;n))) &=\\
fst ((\lambda cn.c\;5\;(c\;3\;(c\;2\;n)))\; fs\; xz) &=\\
fst (fs\;5\;(fs\;3\;(fs\;2\;xz))) &=\\
fst (fs\;5\;(fs\;3\;(fs\;2\;[\;nil\;nil]))) &=\\
fst (fs\;5\;(fs\;3\;((\lambda ep.pair\; (snd\; p)\; (cons\; e\; snd\; p))\;2\;[nil, nil]))) &=\\
fst (fs\;5\;(fs\;3\;[nil, [2]])) &=\\
fst (fs\;5\;[[2], [3, 2]]) &=\\
fst [[3, 2], [5, 3, 2]] & = & [3, 2]
\end{eqnarray*}


\paragraph*{5.1} \textit{Построить терм-пожиратель, то есть такой терм $\mathsf{F}$, что для любого терма $\mathsf{M}$ верно следующее: $\mathtt{FM = F}$}

Пусть $\mathtt{F}$ выглядит как-то так: $\mathtt{F = YX}$, где $\mathtt{X}$ пока неизвестный терм. Из определения fixpoint combinator известно, что $\mathtt{YX = X(YX)}$.

Посмотрим что такое $\mathtt{FM}$: 

\[
\mathtt{FM = (YX)M = (X(YX))M = X(YX)M = XFM = F}
\]

Тогда что такое неизвестный терм $\mathtt{X}$? Это такая функция, принимающая 2 параметра и возвращающая первый. Да это же $\mathtt{K}$!

Answer: $\mathtt{F = YK}$

\paragraph*{5.2} \textit{Построить такой терм $\mathsf{F}$, что для любого терма $\mathsf{M}$ верно следующее: $\mathtt{FM = MF}$}

(позволю себе скопировать :) )

Пусть $\mathtt{F}$ выглядит как-то так: $\mathtt{F = YX}$, где $\mathtt{X}$ пока неизвестный терм. Из определения fixpoint combinator известно, что $\mathtt{YX = X(YX)}$.

Посмотрим что такое $\mathtt{FM}$: 

\[
\mathtt{FM = (YX)M = (X(YX))M = X(YX)M = XFM = MF}
\]

Тогда что такое неизвестный терм $\mathtt{X}$? Это такая функция, принимающая 2 параметра и возвращающая применение второго к первую. Ну запишем это: $\mathtt{X = \lambda xy.yx}$

Answer: $\mathtt{F = Y(\lambda xy.yx)}$

\paragraph*{5.3} \textit{Построить такой терм $\mathsf{F}$, что для любых термов $\mathsf{M}$ и $\mathsf{N}$ верно следующее: $\mathtt{FMN = NF(NMF)}$}

(позволю себе опять скопировать :) )

Пусть $\mathtt{F}$ выглядит как-то так: $\mathtt{F = YX}$, где $\mathtt{X}$ пока неизвестный терм. Из определения fixpoint combinator известно, что $\mathtt{YX = X(YX)}$.

Посмотрим что такое $\mathtt{FMN}$: 

\[
\mathtt{FMN = (YX)MN = (X(YX))MN = X(YX)MN = XFMN = NF(NMF)}
\]

Решаем простейшее уравнение на термы и находим вид неизвестного терма $\mathtt{X}$. Решили: $\mathtt{X = \lambda fmn.nf(nmf)}$

Answer: $\mathtt{F = Y(\lambda fmn.nf(nmf))}$

\paragraph*{6.} \textit{Пусть имеется взаимно-рекурсивное определение функций $\mathsf{f}$ и $\mathsf{g}$: $\mathsf{f = Ffg}$ , $\mathsf{g = Gfg}$. Используя $\mathsf{Y}$-комбинатор, найдите нерекурсивные определения этих функций}

Абстрагируемся:

\[
\mathtt{f = Ffg = (\lambda n.F\;n\;g)f = f'\;f}
\]
\[
\mathtt{g = Gfg = (\lambda n.G\;f\;n)g = g'\;g}
\]

Видно, что и $\mathsf{f}$ и $\mathsf{g}$ есть неподвижные точки для лямбд. Поэтому:

\[
\mathtt{f = Yf'}
\]
\[
\mathtt{g = Yg'}
\]

Функции стали не столько рекурсивные, сколько взаимно-определенными. Продолжаем абстрагироваться:

\[
\mathtt{f = Yf' = Y(\lambda n.F\;n\;g) = Y(\lambda n.F\;n\;(Yg')) = Y(\lambda n.F\;n\;(Y(\lambda n.G\;f\;n))) = Y((\lambda mn.F\;n\;(Y(\lambda n.G\;m\;n)))f)}
\]
Обозначим, для краткости $\mathsf{\lambda mn.F\;n\;(Y(\lambda n.G\;m\;n)) = P}$. Продолжаем абстрагироваться:

\[
\mathtt{f = Y(Pf) = (\lambda s.Y(Ps))f = Y(\lambda s.Y(Ps)) = Y(\lambda s.Y((\lambda mn.F\;n\;(Y(\lambda n.G\;m\;n)))s))}
\]

Все, записываем ответ

\[
\mathtt{f = Y(\lambda s.Y((\lambda mn.F\;n\;(Y(\lambda t.G\;m\;t)))s))}
\]
Для функции $\mathsf{g}$ все аналогично: 

\[
\mathtt{g = Y(\lambda s.Y((\lambda mn.G\;(Y(\lambda t.F\;t\;n))\;m)s))}
\]
\end{document}
