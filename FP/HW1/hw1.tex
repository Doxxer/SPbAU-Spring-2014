\documentclass[russian]{article}
\usepackage[T1]{fontenc}
\usepackage[utf8]{inputenc}
\usepackage{geometry}
\geometry{verbose,tmargin=2cm,bmargin=2cm,lmargin=1cm,rmargin=1cm}
\usepackage{amsmath}
\usepackage{float}
\usepackage{textcomp}
\usepackage{amssymb}
\usepackage{graphicx}
\usepackage{babel}
\usepackage{mathtools}
\usepackage[T2A]{fontenc}

\makeatletter
\@ifundefined{date}{}{\date{}}

\begin{document}

\title{Функциональное программирование. HW\#1}
\author{Тураев Тимур, 504 (SE)}

\maketitle

\paragraph{1} \textit{Найти более короткую версию $\mathtt{iszro}$}
\\

Answer: $\mathtt{iszro \equiv \lambda ntf.\ n(\lambda a.f)t}$\\

$\mathtt{iszro\ 0 \equiv (\lambda ntf.\ n(\lambda a.f)t)\ 0
		\equiv (\lambda ntf.\ n(\lambda a.f)t)\ (\lambda sz.\ z)
		\equiv \lambda tf.\ (\lambda sz.\ z)(\lambda a.\ f)t
		\equiv \lambda tf.\ t \equiv tru}$
		
$\mathtt{iszro\ 1 \equiv (\lambda ntf.\ n(\lambda a.f)t)\ 1
		\equiv (\lambda ntf.\ n(\lambda a.f)t)\ (\lambda sz.\ sz)
		\equiv \lambda tf.\ (\lambda sz.\ sz)(\lambda a.\ f)t
		\equiv \lambda tf.\ (\lambda a.\ f)t
		\equiv \lambda tf.\ f \equiv fls}$

\paragraph{2} \textit{Выразить $\mathtt{plus}$ через $\mathtt{succ}$}
\\

Answer: $\mathtt{plus \equiv \lambda mn.\ m\ succ\ n}$

\begin{align*}
	plus\ 2\ 3 &\equiv (\lambda mn.\ m\ succ\ n)\ 2\ 3 \equiv 2\ succ\ 3 \\
	&\equiv (\lambda sz.\ s(s(z)))\ succ\ (\lambda ab.\ a(a(a(b)))) \\
	&\equiv (\lambda sz.\ s(s(z)))\ (\lambda nxy.\ x(nxy))\ (\lambda ab.\ a(a(a(b))))\\
	&\equiv (\lambda z.\ (\lambda nxy.\ x(nxy))((\lambda nxy.\ x(nxy))\ z) )(\lambda ab.\ a(a(a(b)))) \\
	&\equiv (\lambda nxy.\ x(nxy))((\lambda nxy.\ x(nxy))\ (\lambda ab.\ a(a(a(b)))))\\
	&\equiv (\lambda nxy.\ x(nxy))((\lambda xy.\ x(x(x(x(y))))))\\
	&\equiv \lambda xy.\ x(x(x(x(x(y))))) \equiv 5
\end{align*}

\paragraph{3} \textit{Записать $\mathtt{mult2}$ короче}
\\

$\mathtt{mult2 \equiv \lambda mnsz.\ m(ns)z	\equiv \lambda mns.\ m(ns)}$ (применили $\eta$-преобразование)

\paragraph{4} \textit{Выполнить подстановку}

\begin{itemize}
\item $\mathtt{\lambda yz.\ xyw(zx)\ [x := \lambda y.\ yw]
	\equiv \lambda yz.\ (\lambda y.\ yw)yw(z(\lambda y.\ yw))
	\equiv \lambda yz.\ yww(z(\lambda y.\ yw)) }$

\item Ничего не меняется.

\item $\mathtt{xy(\lambda xz.\ xyz)y\ [y := xz]
	\equiv x(xz)(\lambda x'z'.\ x'(xz)z')(xz) }$

\end{itemize}

\paragraph{5} \textit{Раскрыть скобки}

\begin{itemize}
\item $\mathtt{
	(x(\lambda x.\ ((xy)x))y) \equiv
	x(\lambda x.\ ((xy)x))y \equiv
	x(\lambda x.\ xyx)y
	}$

\item $\mathtt{
	((\lambda p.\ (\lambda q.\ ((q (p r)) s))) ((q (p r)) s)) \equiv
	(\lambda pq.\ q (pr) s) (q (p r) s) \equiv
	(\lambda p'q'.\ q' (p'r) s) (q (p r) s) \equiv
	\lambda q'.\ q' (q (pr) sr) s
	}$
\end{itemize}

\paragraph{6} \textit{Показать}

$\mathtt{
	S(\lambda x.\ M)(\lambda x.\ N) \equiv
	(\lambda fgx.\ fx(gx))(\lambda x.\ M)(\lambda x.\ N) \equiv
	\lambda z.\ (\lambda x.\ M)z((\lambda x.\ N)z) \equiv
	\lambda z.\ MN}$
	
\paragraph{7} \textit{Показать}

\begin{itemize}
\item $\mathtt{
	SKK = (\lambda fgx.\ fx(gx))(\lambda xy.\ x)(\lambda xy.\ x)
	= \lambda x.\ (\lambda xy.\ x)x((\lambda xy.\ x))
	= \lambda x.\ x = I
	}$

\item $\mathtt{
	S(KS)K = (\lambda fgx.\ fx(gx))(KS)K
	= \lambda x.\ (KS)x(Kx)
	= \lambda x.\ S(Kx)
	= \lambda x.\ (\lambda fgz.\ fz(gz))(Kx)
	= \lambda x.\ (\lambda gz.\ (Kxz)(gz))
	= \lambda xgz.\ x(gz)) = B
	}$

\end{itemize}

\paragraph{8} \textit{Степенная функция}\\

$\mathtt{pwr1 \equiv \lambda mn.\ n\ (mult\ m) 1}$ -- по аналогии с $\mathtt{mult1}$ \\

$\mathtt{pwr2 \equiv \lambda mnsz.\ nmsz}$ -- по аналогии с $\mathtt{mult2}$.\\

Оформлять пример долго, но вторую я честно проверил на черновике, она работает.

Кстати, если дважды применить $\eta$-преобразование, будет ли верно, что $\mathtt{pwr2 \equiv \lambda mn.\ nm}$ ?

\end{document}
